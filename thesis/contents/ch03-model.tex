\section{Clickstream and Action-Path Models}
\label{ch:model}

In this chapter, we first formalize few concepts and metrics in clickstream data,
and then describe a proposed clickstream model named \emph{Action-Path model} 
based on recurrent nerual network that models a client side clickstream behavior. 
The action path is slightly different than clickstream since 
a user may uses \emph{back button} or \emph{switch browser tabs} then jumps to visited web pages 
or parallel web pages, say performed a visit action.
A server side clickstream does not containing such detailed level of user clickstream. 
The term \emph{action path} is a generalized concept of clickstream, which replaces individual URLs 
to user actions (with backbutton and browser tab switch effects) in a browser.

For a convinience of discussion, we indiscriminate the use of 
term \emph{action path} and \emph{clickstream} in
this chapter to indicate a series of user actions.

\subsection{Completion Effeciency}

An action path of a visiting session starts from a starting page and ends on a exiting page.
Since we consider the effect of browser back button and browser tab swtich, previous page could
easily be visited twice, if a user clicked the back button. Therefore, a page may directs
to multiple pages. \emph{For instance, an action path could degrade to a linked list if a user
click through to different pages without using back button and switching tabs; or an action
path could become a 1-to-n bipartite graph if a user use back button back to previous page
after clicked a page.}

As a result, we define the \emph{completion effeciency} based on shortest path from starting page
to exiting page, and stay duration of the action path. 

Let an action path is represented on directed cyclic graph, each node represents a visited page,
each edge has a weight that represents the stady duration of its tail node.
Assume the total stay duration of the shortest path from starting page to existing page
is $d_s$, and the total stay duration of the action path is $D$, 
the number of nodes in the shortest path is $n_s$, the total nodes 
in an action path is $N$, the \emph{completion effeciency $E$} is defined 
as follows equation \ref{eqn:completion-effeciency}:

\begin{align}
\label{eqn:completion-effeciency}
\begin{split}
    E = w_1 \frac{n_s}{N} + w_2 \frac{d_s}{D}\\
    w_1 + w_2 = 1
\end{split}
\end{align}

where $w_1$, $w_2$ are hyperparameters to balancing the importance of action path 
and stay duration. According to the duscussion of two special case of action path,
it is easy to prove the range of $E$ is $(0, 1]$. As a complement, we define \emph{zero
completion efficiency} if and only if a user cannot complete a clickstream. Therefore
we have the range of $E$ is $[0, 1]$.

\textbf{Remark 1}. The definition of completion effeciency uses the term of shortest path,
which is the problem of finding a path between the starting page and exiting page
 in a action path (directed cyclic graph) such taht the sum of
the stay duration of its constitunent pages in minimized.
The problem can be solved by Dijkstra's algorithm \cite{dijkstra1959note}.

\textbf{Remark 2}. An action path may increases with more nodes (pages) over time.
The starting page of an action path is always the first page when browser was opened.
However, one can always treat the current visited page is the exiting page due to we
do not know when an user will exit browsing over time. Consequently, $E$ is changing over
browsing.

\subsection{Clickstream Overlap Ratio}



\subsection{URL2Vec Embedding}



\subsection{Action-Path Model}

Recurrent Neural Network (RNN) was describe by Werbos \cite{werbos1990rnn} and 
Rumelhart et al. \cite{Rumelhart:1988:LRB:65669.104451}, the original RNN 
generalize feedforward neural network for sequence based data.

Given a sequence of input $(i_1, i_2, ..., i_T)$, the original RNN computes a
sequence of outputs $(o_1, o_2, ..., o_T)$ by interating the activation function \ref{eqn:bptt}:

\begin{align}
\label{eqn:bptt}
\begin{split}
    o_t = W_{oh} \sigma \left( W_{hi}i_{t} + W_{hh}i_{t-1}\right), t=1,2,...,T
\end{split}
\end{align}

where $\sigma(x) = \frac{1}{1+\exp\{-x\}}$,
and $W_{oh}, W_{hh}, W_{hi}$ are weight parameters between output, hidden and input layers.

The original RNN transfers and maps a sequence to another sequence if and only if the inputs
and the outputs are aligned with equal length. Apparently, the major issue of the original RNN
is the model cannot address a problem if inputs and outputs provided in different length with 
complicated and non-monotonic relationships.

Stutskever et al. \cite{DBLP:journals/corr/SutskeverVL14} present a general end-to-end approach
to sequence learning and estimates the conditional probability of 
$p(o_1, o_2, ..., o_{T'} | i_1, i_2, ..., i_T)$ where $(i_1, i_2, ..., i_T)$ is an input sequence,
$(o_1, o_2, ..., o_{T'})$ is a corresponding output sequence, and $T$ is not required to be equal with $T'$.
Our model convey similar idea from it.

An \emph{Action Path} from user $i$ in session $j$ is consist of 
a sequence of URLs $(U^{ij}_1, U^{ij}_2, ..., U^{ij}_n)$ 
and a sequence of time duration $(d^{ij}_1, d^{ij}_2, ..., d^{ij}_n)$, since each URL 
has a corresponding number that represents the time duration of a user spent on the given page.

\subsection{Action Path Optimization}

$\text{arg}\max_{y} p(o_1, o_2, ..., o_{T'} | i_1, i_2, ..., i_T)$



% - 阐述基于 RNN 的点击流模型,介绍如何通过这个模型对点击流进行预测,这个预测结果包括后续整个点击流的预测,
% - 模型需要能够对整条点击流进行一个评估,这个评估包含了

\cleardoublepage