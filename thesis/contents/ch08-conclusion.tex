\section{Conclusions}
\label{ch:final}

\epigraph{Every age has its own myths and calls them higher truths.}{Anonymous}

% \subsection{Summary}

% 1. what has been done

This thesis proposes an action path model that describes client-side user clickstreams,
which are also known as action paths.
To justify our model, we designed nine browsing tasks for three qualitatively
discussed types of browsing behavior based on the theory of information behavior. We then 
held a user study for these tasks that simulated those behaviors. 
Afterwards, we applied the data collected from user study to our action path model and
analysed the model performance for this data with comparisons to the traditional machine 
learning approach.
We also visualized these data and discovered the common, 
individual, and intersection patterns among the client-side clickstream.
As an application showcase, we illustrated a browser plugin that monitors client-side 
user clickstreams to predict future movements of web browsing. 
We also discussed the benefits of this plugin.
Furthermore, we presented a generic architecture communication flow, the architecture of the plugin, 
as well as the possibilities to standardize the plugin feature in the form of browser 
Web APIs for other developers.

% 2. what are the findings

Our findings answer the research questions that underlie this this thesis:

\paragraph{Understanding} 
(a) A client-side collected clickstream is different from the 
server-side collected clickstream because of parallel visiting 
and multiple website visiting. The three types of suggested browsing behaviors are
goal-oriented, fuzzy and exploring behaviors.
(b) The number of actions, total stay duration and completion efficiency cannot 
provide an accurate classifier for these three behaviors. However, 
the number of actions is more important than the others for indicating 
goal-oriented browsing behavior and the other two features are more important 
for indicating exploring behavior;
(c) The observed patterns in the action path, which include cluster, hesitation, ring, 
star and overlap contribute to different browsing behaviors;
(d) Action paths visually tend to be user-specific but remain common interests 
in goal-oriented behaviors.
\paragraph{Classification} the proposed action path model is 100.00\% accurate for 
the classification of the three types of browsing behavior, which is trained on a user-independent dataset.
\paragraph{Prediction} prediction of three to five future steps can be accurately (>60\%) predicted
in a simplest action path model.

\paragraph{}

% 3. emphasize the importance >> place to large context
Our findings are generic. The model is an action-level model that models
a sequence of user actions and the time of decision making (stay duration). This means 
that it can be used on desktop, and also can be implemented in context of mobile devices, 
or even outside the context of web browsing.
Similar to other user behavior data, a client-side user clickstream or user actions 
directly indicate movements of a user and how they make decisions. Understanding, 
interpreting and predicting these data not only improves the user experience of
web browsing, but also useful to help users to reduce useless browsing, which 
manages their time more efficiently. Moreover, standardizing the data processing process 
can formalize this feature for developers, which will help them to use the behavior predictions to
improve the user experience of their products.

% 4. summarize / conclude
Traditional server collected clickstream data has proved its high value in many 
fields. Our work exposit the value one-step forward, and contributes to models 
and approaches that hope to bring ponderable research to the community and industry.

\cleardoublepage