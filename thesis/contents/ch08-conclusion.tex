\section{Conclusions}
\label{ch:final}

% \subsection{Summary}

% 1. what has been done

This thesis proposed an action path model that describes client-side user clickstream,
as known as action path.
To justify our model, we designed nine browsing tasks for three qualitatively
discussed browsing behavior based on the theory of information behavior, then 
held a user study for these tasks that simulates the behaviors. 
Afterwards, we applied the collected data from user study to our action path model and
analysed the model performance to these data with comparison to traditional machine 
learning approach.
Subsequently, we also visualized these data and closely discovered the common, 
individual and intersection patterns among client-side clickstream.
As an application show case, we illustrated a browser plugin that monitors client-side 
user clickstream to predict future movements of web browsing and discussed the benifits of this plugin.
Futhermore, we presented a generic architecture communication flow and architecture of the plugin, 
as well as the possibilities of standardize the plugin feature as browser Web APIs to other developers.

% 2. what are the findings

Our finding answers the research questions that motivates this thesis:

\begin{itemize}
    \item Understanding: 
        \begin{enumerate}
            \item client-side clickstream is different than server side clickstream because of the existence of
            parallel visiting and multiple website visiting, three suggested browsing behaviros are: goal-oriented, fuzzy and exploring behaviors;
            \item number of actions, total stay duration and completion efficiency cannot provide an accurate classifier for these three behaviors, 
            and the number of actions are more important than others for indication of goal-oriented browsing behavior but other two features are more important for indication of exploring behavior;
            \item the observed patterns in action path including cluster, hesitation, ring, star and overlap contributes to different browsing behaviors;
            \item Action path tend to be user-specific but still contains a small group of common interests in goal-oriented behaviors;
        \end{enumerate}
    \item Classification: the proposed action path model 100.0\% accuratelly classified three browsing behavior;
    \item Prediction: three to five future steps prediction can be accuratelly (>60\%) predicted.
\end{itemize}

% 3. emphasize the importance >> place to large context
Our findings are generic and subservience. The model is an action level model that models
sequence of user actions and time of decision makings (stay duration), which means it
can be use on desktop and also can be implemented in context of a mobile devices, 
or even a outside the context of web browsing.
Similar to other user behavior data, client-side user clickstream or user actions 
directly indicates movements of a user and how they making decisions. Understanding, 
interpreting and predicting these data not only improves the user experience when doing
web browsing, but also useful to help users reducing useless browsing, better controls 
and manages their time. Moreover, by standardize the data processing process can formalize
the feature to developers, and then help them using the behavior predictions to
improve user experience of their products.

% 4. summarize / conclude
Traditional server collected clickstream data has been proved its high value in many 
fields. With our work we exposit the value one-step forward, and contributes to models 
and approaches that hope to bring ponderable research to the community and industry.

\cleardoublepage