\section{Conclusions}
\label{ch:final}

\subsection{Summary}

% 1. what has been done

This thesis proposed an action path model that describes client-side user clickstream.
To justify our model, we designed nine browsing tasks for three qualitatively
discussed browsing behavior based on the theory of information behavior, then 
held a user study for these tasks that simulates the behaviors. 
Afterwards, we applied the collected data from user study to our action path model and
analysed the model performance to these data with comparison to traditional machine 
learning approach.
Subsequently, we also visualized these data and closely discovered the common, 
individual and intersection patterns among client-side clickstream.
As an application show case, we illustrated a browser plugin that monitors client-side 
user clickstream to predict future movements of web browsing and discussed the benifits of this plugin.
Futhermore, we presented a generic architecture communication flow and architecture of the plugin, 
as well as the possibilities of standardize the plugin feature as browser Web APIs to other developers.

% 2. what are the findings

Our finding indicates the completion difficulty in different type of tasks are significant different,
especially the difficulty of fuzzy task is significant difficult than goal-oriented task,
and the goal-oriented task is also significant difficult than exploring task.
For completion efficiency, length of client-side clickstream and total duration of the clickstream

% 3. emphasize the importance >> place to large context
Our findings are generic and subservience. The model can not only be use on desktop but
also can be implemented in a mobilephone, or even a outside the context of web browsing. 
Similar to other user behavior data, client-side user clickstream or user actions 
directly indicates movements of a user and how they making decisions. Understanding, 
interpreting and predicting these data not only improves the user experience when doing
web browsing, but also useful to help users reducing useless browsing, better controls 
and manages their time. Moreover, by standardize the data processing process can formalize
the feature to developers, and then help them using the behavior predictions to
improve their product user experience.

% 4. summarize / conclude
Traditional server collected clickstream data has been proved its high value in many fields. With our work we exposit the value one-step forward, and contributes to models and approaches that hope to bring ponderable research to the community.


\cleardoublepage