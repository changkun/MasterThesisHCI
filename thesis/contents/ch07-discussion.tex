\section{Discussion}
\label{ch:discuss}

We proposed an action path model that models a sequence of user actions
and their decision time of each action simultaneously.
Then we designed and conducted a user study that collects action paths
from participants with different browsing behavior.
We discuss our main findings and the limitations of this work in this chapter.

\subsection{Main Findings}

\paragraph{Clickstream Modeling}

The action path model combines an entire action level clickstream, 
and the stay duration of each action into action path encoder.
Our quantitative results indicate that 
a simply model can easily classify existing three type of browsing
behaviors with 100\% of accuracy, i.e. goal-oriented, fuzzy and exploring.
Even further, the model is able to universally (corss-user) predict 3 to 5 future visit 
page with given 95 percent of browsing context.

\paragraph{Browsing Behaviors and Patterns}

We concluded three browsing behaviors based on information behavior theory that describes
three process of web browsing.
Our qualitative analysis 
first interpret the total number of actions 
are more important to contributes the indication of goal-oriented behavior,
and the toal stay duration and completion efficiency are more important to indicate exploring behavior.

Afterwards, we also observed five patterns from client-side clickstream, the ring and star patterns
appears in fuzzy and exploring tasks, ring pattern is more often in exploring task,
and the star pattern is more often in fuzzy task because of differentiating of information use.
A cluster pattern is an indiccation of an individual intent while browsing, and it may connects
few hesitation pattern.
The overlap pattern discovered in the collected action path gains a low overlap ratio,
which suggests action path tend to be a user-specific behavior but reserve a small region as
common interest in goal-oriented browsing behavior.
Next, an analysis based on Ellis' model and Wilson's theory explose the relationship 
of these patterns to the proposed browsing behaviors, and these patterns partially represents a browsing behavior.
Finally, since the model encodes the entire client-side clickstream and stay duration,
the analysis also explains the reason why our model archieves such a good performance.

\paragraph{\emph{Why the task difficulty is not measured through NASA-TLX?}}

\paragraph{\emph{Why leave-one-out cross validation is not applied in classification task?}}

\subsection{Limitations and Future Works}

\paragraph{Lack of data} 
This thesis has a limitation of the lack of data. 
Though we collected 189 clickstreams from 21 subjects, however, 
comparing to the baseline action path model with 90323 parameters in Chapter \ref{ch:eval},
the dataset is still an tiny dataset for the training and learning task.
Moreover, the validation loss (in Figure \ref{fig:class-loss}) suggests our model remains large capacity to learn more
categories of browsing behavior, and prediction performance may be improved
via reparematrization (in Figure \ref{fig:loss}), 
it is still very interesting to see the performance of our model on a large dataset,
and how the this model can adapts to more informations on the web, such as the topic of 
a page, and interpret more detail with attention mechanism.

\paragraph{Data collection}
Our work simulates three proposed browsing behavior through carefully designed browsing tasks.
This method only limit to a small group of users, which is not an appropariate apparoch for
a large dataset collection.
We planed to conducting a field study that install a clickstream collector while a week, 
however there are only two subjects after our lab study
are willing to participate to the field study.

\paragraph{Reinforcement learning appraoch}
As described in Chapter \ref{ch:model}, the dataset that applies to our action path model
is an action-level dataset, which means the sequence of URLs are essentially an series of 
user actions. This could inspire us to use reinforcement learning appraoch to train
an agent that could explore and learn the environment of the web. Eventually,
the agent will be able to learn and optimize the experience of browsing on the web,
which implicitlly solves the problem of data collection and the lack of supervised data.

\paragraph{Privacy}
This work monitors an action level of clickstream, which stores all browsing history of an
individual person on a thirdparty databases, and hence brings a trust and privacy issue of
the application. We positively argue that this is an trust issue between users and service providers.
As we discussed in Chapter \ref{ch:app}, browser providers collect the data anonymously,
and users use the browser because of trusts, then world wide web consortium formalizes
a standardized web API to developers for using this information, and as a browser user can
either authorize developers to use this API or give an explicit rejection.

\paragraph{Proactive serving} 
We are in the era that intellegent system surrounding us. The way we interact with intellegent system
is not as nature as we interact with other people. 
Communications or interactions between humans in a context does not require any trigger word,
and a person can brush out a needs or reacts to another immediately.
The action path prediction gives an working example that shows proactive serving is possible
if we monitors the environment of web browsing. Therefore, it is interesting to study
how this feature could be used by a user and how users reacts to the elimination of interaction trigger
of an intellegent system.

\cleardoublepage