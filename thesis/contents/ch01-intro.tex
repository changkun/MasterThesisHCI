\section{Introduction}
\label{ch:intro}

\epigraph{That men do not learn very much from the lessons of history is the most important of all the lessons that history has to teach.}{Aldous Huxley}

\subsection{History of Clickstream Research}

% 起源

The word ``clickstream'' \cite{friedman1995} was first coined in 1995 when a media 
article introduced the novel concept of tracing the cyberlife of users over what is currently known
as the ``Internet''. ``clickstream'' contains a sequence of hyperlinks clicked by a 
website user over time. In 1995, the most popular server software 
Apache HTTP \cite{apache1995http} proxy on the Web was developed with a feature that 
recorded the access log of entries. Afterwards, people realized the potential danger and value 
of tracing cyberspace. A major discussion was concluded over clickstream issues, such as 
the frequency-based mining of clickstream \cite{brodwin1995}, privacy concerns
\cite{reidenberg1996governing}, and the database schema of session-based time series data 
\cite{courtheoux2000database}.

% 搜集 clickstream 带来的推动

The privacy discussion concluded that collecting traces over the internet clearly violates 
the rights of users and,
breaches the openness and transparency of a service.
Serious criticism arose that traces damage democratic governance \cite{gindin1997lost}.

Technology is not guilty. Years of discussion have produced, rules 
\cite{reidenberg1996governing} and regulations \cite{skok1999establishing} in cyberspace,
means of protecting information privacy in cyberspace transactions \cite{kang1997information},
and approaches to resolve conflicting issues in international data privacy \cite{reidenberg1999resolving}.

Meanwhile, bussinessmen have agilely reasponded to the concept of collecting clickstream 
and immediately initiated commercial tracking of their customers to measure product success \cite{schonberg2000measuring}
and improve marketing effects \cite{novick1995}, customer service, and precise advertisements\cite{reagle1999platform, bucklin2000sticky}.

At the turn of this century, there has been common acceptance of the technology of clickstream.
Clickstream data has been confirmed by industrial practice, which has opened up a new era in 
customer service \cite{walsh2000internet}. Most websites' users habe begun to accept that 
their click path data will be aggregated and analysed on the server side \cite{carr2000hypermediation}.

Clickstream data grows and becomes plentiful quickly. Researchers have begun to convey 
track customer selections, which it the original idea behind of clickstream,
into various applications, such as usability testing \cite{Waterson:2002:LOW:506443.506602}
and understanding social network sentiment \cite{Schneider:2009:UOS:1644893.1644899}.
Researchers have also developed visualizing
techniques to better interpret clickstream data \cite{Waterson:2002:DTU:1556262.1556276}.

Analysis, reports, and characterizing of clickstream have gained in popularity. 
Mobasher et al. \cite{Mobasher:2001:EPB:502932.502935}
have suggested personalizing users based on association rules from their web usage data. 
Chatterjee et al. \cite{chatterjee2003modeling} 
have first proposed that e-commerce websites should use clickstream instead of essential choice 
to track customer navigation patterns, thereby 
associating and binding products for the observing responses of a customer.

As characterization and the understanding of behavior based on clickstream data have become popular, 
more research have proposed methods to understand server clickstream data.
Padmanabhan et al. \cite{Padmanabhan:2001:PID:502512.502535} have
proposed an algorithm to address personalization from incomplete clickstream data, which implies
the security problem potential of a potential information leak from clickstream data. 
Regarding search engine indexing, Lourenco at al. \cite{Lourenco:2006:CWC:1145581.1145634} recommends an approach for
the detection and containment of web crawlers based on server-side recorded visiting log files.

A short review of clickstream history has indicated that almost all research have formulated their 
methodology based on server-recorded clickstream data.
A daily user is always allowed simultaneous accesses to parallel pages and windows and may even be allowed,
to switch across multiple websites for a browsing purpose.
An obvious missing aspect of those papers is that server-recorded data tends to be incomplete for 
characterizing a visiting user, and the log data can only applied to on a specific website. 
Our research no longer serves the server-side clickstream; instead, it
focuses and contributes to client-side collected clickstream data 
for a real visiting session of a user in a browser.

\subsection{This Thesis}

% 大体上介绍本论文想要研究 clickstream 的内容,
% 这包括如何开展 clickstream 数据的搜集工作,搜集任务是什么,主要使用的方法,
% 以及得出的结论。根据这些结论,文章提出了一个客户端的插件,
% 能够在现代浏览器上支持这样的预测,
% 同时还进一步探讨了此项功能作为浏览器内建功能甚至浏览器 API 的可能性。

% 本文将视线移出服务端点击流数据,将焦点转移到客户端点击流数据,尝试对客户端独立用户的点击流进行解释。
% 本thesis 由以下几个部分组成。

The main part of the thesis is structured in different chapters, and answers the following
three research questions:

\begin{enumerate}
    \item \textbf{Understanding}: Why does collecting clickstream on the client-side differ 
        from on the server-side?
        What are the most significant and identifiable user behaviors and activity patterns 
        that can be observed or algorithmically detected in the context of web browsing that 
        indicate information needs?
        Which form of quantitative data can characterize a definitive boundary for 
        distinguishing a user's browsing behaviors?
    \item \textbf{Classification}: How accurately or affirmatively can we progressively model 
        or identify the proposed browsing behaviors that makes an intelligent system 
        serve proactively?
    \item \textbf{Prediction}: How many future movements of a user can be accurately inferred 
        from the context of web browsing, and how much context is required for the prediction?

\end{enumerate}

% 第二章讨论了目前已有的客户端点击流研究。
Chapter \ref{ch:relate} first discusses the existing user behavior research based on clickstream data. 
Next, the chapter discusses the evolution of the theory regarding information seeking behavior 
as our experiment's foundation.
In addition, we summarize the reason for the recent increase of the neural approach 
in different scientific area and the state-of-the-art direction for sequence learning, 
that have been proposed in neural network research.
% 第三章介绍了我们的数据类型、模型的设计。
Chapter \ref{ch:model} defines the completion efficiency of a clickstream. Next, we
formalize our proposed sequence to sequence encoder and decoder models for client-side
clickstreams. We also present the training techniques for the proposed model.
% 第四章详细描述了实验的设计,介绍了每个所设计任务的原因和考虑,讨论了每个所设计的任务中潜在的数据的搜集和分析工作。
In Chapter \ref{ch:exp}, we present our experiment for a lab study
and construe the design justification for context-given web browsing tasks for our subjects 
based on information behavior theory.
% 第五章分析了第四章描述中搜集到的数据,基于 SVM、t-SNE 分析了除了点击流之外的几个特征,定量分析,得出了结论。。。。
% 然后基于第三章中提出的模型,分析了。。。。的结果,得出了。。。的结论。
In Chapter \ref{ch:eval},
we conduct a quantitative analysis with described data from our lab study, based on support
vector machine (SVM), t-SNE, and our proposed action path model.
The evaluation produces very promising results.
% 最后对用户在任务下的点击流进行了可视化,定性分析,进一步讨论了用户的行为,从而进一步验证了我们的结论。。。
Moreover, we visualize the clickstream through a directed graph by combining our training model outputs.
We also perform a qualitative analysis on all clickstreams, and the analysis provides evidence 
that further verifies the correctness of our model.
% 第六章首先介绍了浏览器插件的工作原理,讨论了我们插件的工作架构。介绍了插件的客户端和服务端的实现方案。
Chapter \ref{ch:app} describes how we developed a browser plugin 
for Google Chrome as a possible application for our model. The plugin can accurately predict 
the next possible visiting pages of a user. 
In addition, we generalize the design of our plugin architecture between the client and server.
We also discuss the possibilities for that architecture to serve as a standard web API 
for web developers.

% 第七章对整个 thesis 进行了总结,讨论了本文工作中存在的不足以及未来可改进的方向。
In the last two chapters, we discuss this thesis' decisions, the limitations of this work, and
summarize the findings of our thesis, and any the possible future improvements and 
directions for research.
\cleardoublepage