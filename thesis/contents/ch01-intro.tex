\section{Introduction}
\label{ch:intro}

\subsection{The Origin of Clickstream Research}

% 讨论 clickstream 的起源,讨论 clickstream 的影响,
% 为什么 clickstream 变得重要,为什么 clickstream 是一个值得研究的类别,
% clickstream 的价值都体现在哪些方面,以前的一些 clickstream 都有些什么内容。

% 起源

The word "clickstream" \cite{friedman1995} was first coined in 1995, a media comments article 
introduced a novel concept of tracing cyberlife of users over the nowadays "Internet". Informally,
a "clickstream" contains a sequence of hyperlinks clicked by a website user over time.
At the same year, the most popular server software Apache HTTP \cite{apache1995http} proxy on the Web was developed with a feature
that records access log of entries.
Afterwards, people realized the potential danger and value of tracing cyberspace, which  a
large discussion of clickstream influences, such as frequency based mining of clickstream \cite{brodwin1995},
privacy concerns \cite{reidenberg1996governing}, and database schema of session based time series data \cite{courtheoux2000database}.

% 搜集 clickstream 带来的

Privacy discussion concludes collecting traces over net clearly offence the rights of users,
the practice violates the openness and transparency of a service to a user.
Serious criticism arise the tracing becomes a loss of democratic governance \cite{gindin1997lost}.

Technologies is not guilty. After years of discussion, positive opinion proposes the rules 
\cite{reidenberg1996governing} and regulations \cite{skok1999establishing} in cyberspace,
means of protecting information privacy in cyberspace transactions \cite{kang1997information},
and approaches to resolve conflicting international data privacy \cite{reidenberg1999resolving}.

Meanwhile, bussiness man agilely responses to the concept and immediately initate 
commercial tracking of their customer to improving marketing affects \cite{novick1995}, 
customer service and precise advertisment\cite{reagle1999platform, bucklin2000sticky}, 
even measuring product success \cite{schonberg2000measuring}.

At the turn of this century, common reviews start accept the technology of clickstream,
clickstream data has confirmed by industrial practice, which opens a new era in 
customer service \cite{walsh2000internet}, most of website users start accept their click path data 
be aggregate analysed on the server side \cite{carr2000hypermediation}.

Clickstream data grows fast and becomes plentiful, researchers start convey the original concept of clickstream,
tracking customer selections, into various applications, such as usability testing \cite{Waterson:2002:LOW:506443.506602},
understanding social network sentiment \cite{Schneider:2009:UOS:1644893.1644899}, and developed visualizing
technique to better interpret clickstream data \cite{Waterson:2002:DTU:1556262.1556276}.

Analysis, reports and characterizing of clickstream gains its popularity, Mobasher et al. \cite{Mobasher:2001:EPB:502932.502935}
suggests personalize user based on association rule from their web usage data. Chatterjee et al. \cite{chatterjee2003modeling} 
first proposed E-commerce websites should use clickstream to tracking customer navigation pattern instead of essential choice, 
associating and binding products for observing responses of a customer.

With the arise of characterizing and behavior understanding on clickstream data, more and more
research proposes methods for the understanding of given server clickstream data.
Padmanabhan et al. \cite{Padmanabhan:2001:PID:502512.502535}
proposed an algorithm to address personalization from incomplete clickstream data, which implies
a security problem potential information leak from clickstream data. 
Moreover, affected by search engine indexing, Lourenco at al. \cite{Lourenco:2006:CWC:1145581.1145634} recommends an approach for
the detection and containment of web crawler based on server side recorded visiting log file.

After a short review of clickstream history, almost all research putforwards their method 
based on server recorded clickstream data. Note that a daily user is always allowed accesses parallel pages simultaneously and even switching across multiple websites for 
a browsing purpose.
An obvious missing aspect of those papers is the server log data is incomplete to 
a characterizing visited user, and the log data only appropriate for a specific website. 
As an observation, our research no longer surves server side clickstream, 
but focus and contributes to a client side collected clickstream data for real visiting session of a user in a browser.

\subsection{This Thesis}

% 大体上介绍本论文想要研究 clickstream 的内容,
% 这包括如何开展 clickstream 数据的搜集工作,搜集任务是什么,主要使用的方法,
% 以及得出的结论。根据这些结论,文章提出了一个客户端的插件,
% 能够在现代浏览器上支持这样的预测,
% 同时还进一步探讨了此项功能作为浏览器内建功能甚至浏览器 API 的可能性。

% 本文将视线移出服务端点击流数据,将焦点转移到客户端点击流数据,尝试对客户端独立用户的点击流进行解释。
% 本thesis 由以下几个部分组成。

The main part of the thesis is structured in different chapters.
% 第二章讨论了目前已有的客户端点击流研究。
Chapter \ref{ch:relate} discusses the exitsting user behavior research based on clickstream data firstly.
Then we summaried the reason of recent raise of neural approach in different scientific area and the 
state-of-the-art approaches for generic sequence learning, whose proposed in neural network research.
% 第三章介绍了我们的数据类型、模型的设计。
Chapter \ref{ch:model} first defined the completion effeciency of a clickstream, then we
formalizes our proposed sequence to sequence encoder/decoder model for client side
clickstream as well as the training techniques for the proposed model.
% 第四章详细描述了实验的设计,介绍了每个所设计任务的原因和考虑,讨论了每个所设计的任务中潜在的数据的搜集和分析工作。
In subsequent chapter, chapter \ref{ch:exp}, we present our experiment for a lab study,
and construe the design reason of context given web browsing tasks for our subjects.
% 第五章分析了第四章描述中搜集到的数据,基于 SVM、t-SNE 分析了除了点击流之外的几个特征,定量分析,得出了结论。。。。
% 然后基于第三章中提出的模型,分析了。。。。的结果,得出了。。。的结论。
Afterwards, in chapter \ref{ch:eval}, based on SVM, t-SNE and our proposed model, 
we conducte a quantitative analysis with described data from our lab study, 
the evaluation shows a very promising result and the result suggests TODO:. % 还要介绍评估得分。
% 最后对用户在任务下的点击流进行了可视化,定性分析,进一步讨论了用户的行为,从而进一步验证了我们的结论。。。
Moreover, we visualizes the clickstream through directed graph, by combining our traning model outputs,
we also performs a qualitative analysis to all graphs, the analysis gives evidences that further 
verified the correctness of our model.
% 第六章首先介绍了浏览器插件的工作原理,讨论了我们插件的工作架构。介绍了插件的客户端和服务端的实现方案。
In chapter \ref{ch:app}, as a consequence of our analysis, we developed a browser plugin 
for Google Chrome as a possible application to our model. The plugin can fairly predict 
the next possible visiting pages of a user. In addition, we generalize the design of
our plugin architecture into a communication protocol between client and server,
and then the possibilities to being a standard Web API to developers.
% 第七章对整个 thesis 进行了总结,讨论了本文工作中存在的不足以及未来可改进的方向。
To conclude, we finally summarize the findings of our thesis, 
the existing drawbacks of our study, as well as the possible 
future improvements and directions of the thesis in the final chapter.
\cleardoublepage