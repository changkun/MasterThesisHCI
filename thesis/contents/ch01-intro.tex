\section{Introduction}

\subsection{Origin of Clickstream Research}

The word "clickstream" first coined in 1995 \cite{friedman1995}, a media comments article 
introduces a novel concept of tracing cyberlife of users over the nowadays "Internet". 
Afterwards, people realized the potential danger and value of tracing cyberspace, which opens a
large discussion of clickstream influences, such as frequency based mining of clickstream \cite{brodwin1995},
privacy concerns \cite{reidenberg1996governing}, and database schema of such a time series data \cite{courtheoux2000database}.

Privacy discussion concludes collecting traces over net clearly offence the rights of users,
the practice violates the openness and transparency of a service to a user.
Serious criticism arise the tracing becomes a loss of democratic governance \cite{gindin1997lost}.

Technologies is not guilty. After years of discussion, positive opinion proposes the rules 
\cite{reidenberg1996governing} and regulations \cite{skok1999establishing} in cyberspace,
means of protecting information privacy in cyberspace transactions \cite{kang1997information},
and approaches to resolve conflicting international data privacy \cite{reidenberg1999resolving}.

Subsequently, bussiness man agilely responses to the concept and immediately initate 
commercial tracking of their customer to improving marketing affects \cite{novick1995}, 
customer service and precise advertisment\cite{reagle1999platform, bucklin2000sticky}, 
even measuring product success \cite{schonberg2000measuring}.

At the turn of this century, common reviews start accept the technology of clickstream,
clickstream data has confirmed by industrial practice, which opens a new era in 
customer service \cite{walsh2000internet}, most of website users start accept their click path data 
be aggregate analysed on the server side \cite{carr2000hypermediation}.

Clickstream data grows fast and becomes plentiful, researchers start convey the original concept of clickstream,
tracking customer selections, into various applications, such as usability testing \cite{Waterson:2002:LOW:506443.506602},
understanding social network sentiment \cite{Schneider:2009:UOS:1644893.1644899}, and developed visualizing
technique to better interpret clickstream data \cite{Waterson:2002:DTU:1556262.1556276}.

Analysis, reports and characterizing of clickstream gains its popularity, Mobasher et al. \cite{Mobasher:2001:EPB:502932.502935}
suggests personalize user based on association rule from their web usage data. Chatterjee et al. \cite{chatterjee2003modeling} 
first proposed E-commerce websites should use clickstream to tracking customer navigation pattern instead of essential choice, 
associating and binding products for observing responses of a customer.

\subsection{This thesis}

% 讨论 clickstream 的起源,讨论 clickstream 的影响,
% 为什么 clickstream 变得重要,为什么 clickstream 是一个值得研究的类别,
% clickstream 的价值都体现在哪些方面,以前的一些 clickstream 都有些什么内容。

% 大体上介绍本论文想要研究 clickstream 的内容,
% 这包括如何开展 clickstream 数据的搜集工作,搜集任务是什么,主要使用的方法,
% 以及得出的结论。根据这些结论,文章提出了一个客户端的插件,
% 能够在现代浏览器上支持这样的预测,
% 同时还进一步探讨了此项功能作为浏览器内建功能甚至浏览器 API 的可能性。

\cleardoublepage