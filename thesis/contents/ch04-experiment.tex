\section{Experiment}
\label{ch:exp}

\epigraph{We must know. We will know.}{David Hilbert}

In this chapter, we rationalize the process of our lab study based on the theory of
human information behavior, then construe the purpose of context given web browsing tasks 
to our subjects.

The lab study took place during the last two weeks of November, from 14/11/2018 to 29/11/2018
in Frauenlobstrasse 7a, a faculty building of Ludwig-Maximillians-Universitaet Muenchen.
Our action path data was collected by a self-developed embedded collector plugin installed in 
the mainstream browser, i.e. Google Chrome, on a self-provided desktop computer and a laptop.

In the lab study, we select three mainstream websites, Amazon/Medium/Dribbble, 
that covers categories for shopping, media consuming and design brainstorming with design 
reasons (discuss later in Section \ref{sec:task-design}).
Then we manually designed 35 reasonable tasks and finally selected 9 
context-given browsing tasks (three for each website, discuss in 
Section \ref{sec:task-design}) to simulate three different proposed browsing behavior,
namely \emph{goal-oriented, fuzzy and exploring behaviors}.

Each task requires participant start from a starting page of a given website, and
all tasks do not restrict participants only use the given website, but also allow they 
access websites outside the domain of landing page to help they complete the task 
(this offer is explicitly informed to participants before participation).
Participants start browsing after they completely understand the requirements of 
each task, and no interruption or question answering during the task
except exceeding the time limit of a task, however, subjects can either acquire more time 
to accomplish the task or give up directly when feeling a task are too difficult to he or she.

The study is designed as a within-subject study, thus every participant performs all tasks. 
To eliminate the learning effect due to a long time of using same websites, 
we use Latin square \cite{cochran1950experimental} 
for the device (Desktop/Laptop) and tasks participation order to our subjects.

Our lab study obtained 21 participants with a mean age of 23.04 (standard deviation of 3.216, 
min=18 and max=19) took part in the study, 10 male and 11 female, who are recruit 
anonymously and randomly arrived via a mailing list.

\subsection{Environment}

The lab study uses two self-provided devices: a desktop computer and a mobile laptop.
The reason of choose two morphology of computing device is our study requires recording
a complete clickstream of during the study.

A major issue of mobile devices is the operating system installed on aa mobile phone does not
authorize the permission of allowance to collect data precisely over pages or user actions.
Though Android device can overpass system permission to privilege, the user behavior 
between iOS and Android device has different personalities \cite{sandoio2018android}, 
and subjects shows \cite{reinfelder2014androidios} abnormal awareness behavior 
regarding security and privacy issues when handling a newly provided Android device 
after the switch from an iOS device.
Therefore, to eliminate this awareness, we stick our study environment to desktop devices,
which empower us easily collects the clickstream data from browsers with plugin supports.

Although all modern browsers support plugin development, considering the usage share of 
all browsers on the market, Google Chrome \cite{wiki2018share} obtains 61.7\% 
market shares of desktop browsers, and Apple Safari only shares
15.0\% of the market. Google Chrome dominant market of the desktop web browser.

Hence, we decide to use Chrome to establish our plugin for data collection.
The questionnaire after our lab study indicates the browsers usage share of all subjects, 
as shown in Table \ref{table:sharesubjects}, which further supports our decision of 
browser selection.

\begin{table}[H]
    \small
    \centering
    \setlength{\belowcaptionskip}{10pt}
    \caption{Browser Usage Shares of Lab Study Subjects}
    \begin{tabular}{ccccc}
          \toprule
        & \textbf{Google Chrome} & \textbf{Apple Safari} & \textbf{Mozilla Firefox} & \textbf{Microsoft Edge} \\
          \hline
          Number     & 11 & 5 & 3 & 2 \\
          Percentage & 52.38\% & 23.81\% & 14.29\% & 9.52\% \\
          \bottomrule
    \end{tabular}
    \label{table:sharesubjects}
\end{table}

% \subsection{Context-given Websites}

% In our study, we gathered various websites covering most of the categories that people do
% on the Internet: Social Networks, Shopping, Email, Media Consuming, Search, Production 
% and etc \cite{lori2018internet, lori2017internet}. All selected website are listed 
% in Table \ref{table:websites}.

% \begin{enumerate}
%     \item social network: facebook / twitter / weibo
%     \item shopping: amazon.com / ebay.com
%     \item search: google.com / bing.com / baidu.com
%     \item media: medium.com
%     \item development: github.com
%     \item design dribbble.com
%     \item study medien.ifi.lmu.de
%     \item study en.uni-muenchen.de
%     \item streaming media youtube.com
%     \item research arxiv.com
%     \item study ielts.org
%     \item media bloomberg.com/europe
%     \item social community reddit.com
% \end{enumerate}

% \begin{table}[H]
%     \small
%     \centering
%     \setlength{\belowcaptionskip}{10pt}
%     \caption{Browser Usage Shares of Lab Study Subjects}
%     \begin{tabular}{ccccc}
%           \toprule
%         & \textbf{Google Chrome} & \textbf{Apple Safari} & \textbf{Mozilla Firefox} & \textbf{Microsoft Edge} \\
%           \hline
%           Number     & 11 & 5 & 3 & 2 \\
%           Percentage & 52.38\% & 23.81\% & 14.29\% & 9.52\% \\
%           \bottomrule
%     \end{tabular}
%     \label{table:sharesubjects}
% \end{table}

% \subsection{Pilot Study}

% Initially, we developed a web crawler that downloaded the entire medien computer science website 
% \footnote{\url{https://medien.ifi.lmu.de}}
% .... TODO discuss how we go here?


\subsection{Browsing Behaviors}
\label{sec:behavior}

Before we explain the design reason for our context-given browsing task, 
we first present and discusses three types of user browsing behavior: 
\textbf{goal-oriented}, \textbf{exploring} and \textbf{fuzzy}.

These three terminologies are aggregated and incorporated from behaviors that concluded 
in former qualitative research for web browsing behavior, 
all these terminologies are based on a fundamental theory of 
interdisciplinary perspective information seeking behavior \cite{wilson1997information},
which was discussed in Section \ref{sec:info-seek}.
Table \ref{table:info-seek} compares the terminology differences between former research 
and our thesis.

\begin{table}[H]
    \small
    \centering
    % \setlength{\belowcaptionskip}{10pt}
    \caption{Terminologies comparison of information behavior on the web}
    \begin{adjustbox}{width=\textwidth}
        \begin{tabular}{ccccc}
            \toprule
            \textbf{Author} & \textbf{Terminologies} & \textbf{Terminologies} & \textbf{Terminologies} & \textbf{Main Factors} \\
            \hline
            \cite{choo1999information} & Formal search & \makecell{Conditioned viewing; \\ Informal search} & Undirected viewing & \makecell{Psychological; demographic;\\ role-related environmental; \\source characteristics} \\
            \cite{johnson2017patterns} & \makecell{Directed browsing; \\Known-item search} & \makecell{Semi-directed browsing; \\Explorative seeking; \\``You do not know what you need''; \\Re-finding} & Undirected Browsing & Behavior \\
            \textbf{This thesis} & \textbf{Goal-oriented} & \textbf{Fuzzy} & \textbf{Exploring} & \textbf{Purpose} \\
            \bottomrule
        \end{tabular}
        \label{table:info-seek}
    \end{adjustbox}
\end{table}

To justify our terminology, we combine the six qualitative activities from Ellis' Model 
\cite{ellis1989behavioural} and ``information use'' from Wilson's framework 
\cite{wilson1997information} in information behavior theory 
to represent our summarized browsing behaviors:

\paragraph{Goal-oriented behavior} \emph{occurs when a user initiates 
visiting session on the web caused by a determined objective in a specific context}, 
such as business work, 
social communication, university study, literature research, etc. 

Goal-oriented behavior indicates a piece of active information behavior.
Instead of \emph{formal search}, that only covers the phase of ``monitoring'' and ``extracting''
(or \emph{directed browsing} and \emph{known-item search} that covers ``browsing'' and ``differentiating'' or 
``monitoring'' and ``extracting'' respectively), 
goal-oriented browsing behavior contains the entire life cycle of human information behavior starts
from ``starting'' phase. By observing a browsing behavior, a determined ``information use'' 
can be overserved and concluded.

For instance, a college student intentionally need a latest lecture slide (\emph{information use} observed), 
the student then opens web browser, access college website (\emph{starting}) and navigates to the lecture homepage 
(\emph{chaining}, \emph{browsing}, and \emph{differentiating}).
Finally, the student exit browsing after download the slides (\emph{monitoring} and \emph{extracting}).

\paragraph{Exploring behavior} \emph{occurs when a user initiates browsing session 
aimlessly with no clear observed extracting or information use during the session, 
the person greedily or breadth-first consumes and the content on the Web without 
any information extracting and information use}, such as media consuming, learning 
before using and etc.

Exploring browsing behavior indicates a complete opposite behavior comparing to 
goal-oriented borwsing behavior. More formally describing exploring behavior using Ellis' model, 
the behavior represents ``chaining'' and ``browsing''
without ``differentiating'' and ``extracting'' from ``starting'' while information seeking.

For instance, a person who accesses an unknown utility web application (\emph{starting}), 
he or she explores what functions are provided one by one and 
what he or she can do while using the application (\emph{chaining} and \emph{browsing}).

\paragraph{Fuzzy behavior} \emph{occurs when a user initiate visiting session for information use
 with non-systematic, incomplete prior knowledge that may browsing ongoing for updating 
the framework of knowledge until final acquisition or abandon.}

Fuzzy behavior in a browsing behavior in between of goal-oriented and exploring behaviors.
Instead of only ``chaining'' and ``browsing'' from ``starting'', fuzzy behavior also do
``differentiating'' or ``monitoring'' while information seeking.

For instance, a researcher heared a new technique proposed in another scientific field 
that may influence he or she's research, then the person 
opens a search engine (\emph{starting} and \emph{chaining}) to seek (\emph{browsing}) 
existing (\emph{differentiating}) follow up researches (\emph{monitoring}). 
The browsing may ends without information use because of the technique is irrelevant 
to he or she's research.

\paragraph{Remark} Table \ref{table:ellis} illustrates the existence of activities of our three
browsing behavior. Note that ``information needs'' is not suggested in Wilson's theory \cite{wilson1981user} 
since ``information needs'' can not be clearly observed beforehand information seeking but sometimes
observed after information use. Therefore we do not take information need into the 
consideration of our terminologies.

\begin{table}[H]
    \small
    \centering
    % \setlength{\belowcaptionskip}{10pt}
    \caption{Existence of activities from Ellis' Model and information use in 
    goal-oriented, exploring and fuzzy browsing behavior}
    \begin{adjustbox}{width=\textwidth}
        \begin{tabular}{ccccccccc}
            \toprule
            \multicolumn{1}{c}{\multirow{2}{*}{\textbf{Behaviors}}}  & \multicolumn{1}{c}{\multirow{2}{*}{\textbf{Information Need}}} & \multicolumn{6}{c}{\textbf{Information Seeking}}                                    & \multicolumn{1}{c}{\multirow{2}{*}{\textbf{Information Use}}} \\ \cline{3-8}
            \multicolumn{1}{c}{}                                     & \multicolumn{1}{c}{}                                  & \textbf{Starting} & \textbf{Chaining} & \textbf{Browsing} & \textbf{Differentiating} & \textbf{Monitoring} & \textbf{Extracting} & \multicolumn{1}{c}{}  \\
            \hline
            Goal-oriented                                            &   N/A                                                 &     Exist         &      Exist        &   Exist           &        Exist               &       Exist           &        Exist          &      Exist              \\
            Fuzzy                                                &   N/A                                                 &     Exist         &      Exist        &   Exist           &        Exist               &       Exist           &                     &                       \\
            Exploring                                                    &   N/A                                                 &     Exist         &      Exist        &   Exist           &                          &                     &                     &                       \\
            \bottomrule
        \end{tabular}
        \label{table:ellis}
    \end{adjustbox}
\end{table}

\subsection{Tasks Design}
\label{sec:task-design}

We designed 35 browsing tasks, after conduct a pilot study, 
9 tasks are selected for three websites: Amazon.com, Medium.com and Dribbble.com 
because of the following reasons:

\begin{enumerate}
    \item These three websites all have coresponding tasks to the three type of browsing behavior;
    \item Each of the task can be finished around 5 to 10 minutes;
    \item All these websites are mainstream websites, they do not require 
        massive professional domain knowledge for using.
\end{enumerate}

In addition, the unselected tasks are listed in Appendix \ref{appendix:unselected}.

\subsubsection{Tasks of Goal-oriented Behavior}

We designed and selected an appropriate goal-oriented task for selected websites respectively,
and each task is designed with three specifically designed information need as
the cause of information use.

\paragraph{Amazon.com} \emph{Assume your smartphone was broken and you have 1200 euros 
    as your budget. You want to buy an iPhone, a protection case, and a wireless 
    charging dock. Look for these items and add them to your cart.}

This task initiate from the homepage of Amazon (\emph{starting} and \emph{chaining}), 
and it contains three determined objective since 
a subject is required to add three specific items to the cart (\emph{information use}). 
There are few hidden consideration behind the task (\emph{browsing} and \emph{differentiating}),
which makes the task more realistic (\emph{monitoring} and \emph{extracting}): 
a) There is a budget of this task, which requires subjects must consider the
price of items instead of simply add the first recommended item to cart; 
b) the starting page is amazon.com instead of amazon.de. This decision requires
subjects must also consider the currency rate between US dollars and Euros for budget.
c) There are some items cannot be shipped to Germany (the study took place in Germany).
As a result, subjects cannot add these items to cart and they should find other alternatives.

\paragraph{Medium.com} \emph{Assume you were making plans for your summer vacation. 
        You want to visit Tokyo, Kyoto, and Osaka. You want to find out what kind of experience other people made 
        when traveling to these three places in Japan. Your task is to find three posts 
        for traveling tips regarding these cities. Elevate a post if it is one of your choices.}

This task contains three determined purpose since there are three fixed traveling 
destination (\emph{extracting} and \emph{information use}).
The task also implies few considerations that increase the required interaction of 
the task to subjects:
a) The website only offers English version, some Japanese character may appear in an article,
thus, a translation util may be used while the study (\emph{starting} and \emph{chaining});
b) An article may apears numerous noun, such as toponym. Search engine may used while the study (\emph{browsing});
c) the articles, those require a membership to unlock reading, cannot be elevated (\emph{differentiating}). 

\paragraph{Dribbble.com} \emph{You are hired to a Cloud Computing startup company. You get an assignment to 
        designing the logo of the company. Search for existing logos for inspiration and 
        download three candidate logos you like the most.}

The task also has three determined prupose since subjects are quired to download three candicate trademarks (\emph{extracting} and \emph{information use}).
While the participation, subjects still need take few implicit facts in to account:
a) Subjects who unfamiliar with the term "Cloud Computing" need visit other explainations to figure out
the vision and mission of this type of company (\emph{starting}), and subjects whom already familiar with the term
still need to compares the designed made by other competitors (\emph{chaining}, \emph{browsing} and \emph{differentiating}).
b) Subjects should aware some of the designs shared on the website are not suitable for trademark or icon design (\emph{monitoring}).

\subsubsection{Tasks of Exploring Behavior}

Exploring tasks simply do not provides any deterministic objective,
and all websites has a designed exploring task for subjects.

\paragraph{Amazon.com} \emph{Look for a product category that you are interested in and start browsing. 
        Add three items to your cart that you would like to buy.}

Although the task do not require any specific items to the subjects, the task remains three different
purpose because participants need add three items to the cart. This designed task 
is aimlessly because: all tasks is not specifically informed to participants, 
they either do not have needs of buying items or 
formerly exist needs of buying a specific category but do not have a product candicate yet.
Besides, the description of the task ask participants start from a product category (\emph{starting} and \emph{chaining}), which avoids 
goal-oriented buying a specific product.

\paragraph{Medium.com} \emph{Visit a category you are interested in and elevate three post you like.}

Similar reason as discussed in Amazon.com's exploring task (\emph{starting} and \emph{chaining}). 
It is well to be remined that Medium is a media
website, visiting a specific article formerly read before participation is relatively difficult 
since all contents showed to users are daily updated. Thus the task can be directly consider as an exploring task.

\paragraph{Dribbble.com} \emph{Explore dribbble and download three images you like the most while you browse.}

Dribbble illustrates designs by using image gallery (\emph{starting} and \emph{chaining}). 
The major difference between Dribbble and Google Image Search
is that dribbble is a user-centered content aggregation website, however Google Image Search is a simple content aggregation engine.
As a result, there will be two different interaction in Dribbble: exploring designs based on keywords and categories,
or exploring designs based on users. The latter helps its user finding similar designing style.
The task is aimlessly since the task simply describes nothing and completely let participants explore their preferences
of design styles.

\subsubsection{Tasks of Fuzzy Behavior}

Each of our selected websites also has an fuzzy task respectively, and there are three major goals per task
as control condition to subject action path in our experiment.

\paragraph{Amazon.com} \emph{You want to buy a gift for your best friend as a birthday present.
        Add three items to your cart as candidate.}

The clearness of the task is stronger than exploring task but weaker than goal-oriented task, because
The task restricts participants adding items for a specific purpose (birthday present) but not points
any specific product (no \emph{extracting}).

\paragraph{Medium.com} \emph{Assume you got an occasion to visit China for business. 
        You are free to travel to China for a week. 
        You want to make a travel plan for touring China within a week. Your task is to find out what kind 
        of experience other how people made when going to secondary cities or towns in China, then decide 
        on three cities you want to visit (excluding  Beijing, Shanghai, Guangzhou, and Shenzhen). 
        Elevate if a post helped you make a decision.}

The clearness of the task is stronger than exploring task, because it asks a participant 
to exploring a non-deterministic direction of looking for secondary cities (no \emph{extracting}).
But the clearness of the task is weaker than goal-oriented task due to secondary cities described
in Medium's user posts is unclear, participants suppose to make decision themselves.
Furthermore, this ask is asking regarding traveling China around a week. Cities cannot be randomly
selected because to make traveling plan requires consider geographic location of the city.

\paragraph{Dribbble.com} \emph{You are preparing a presentation and need one picture for each of these animals: 
    cat, dog, and ant. Download the three pictures you like the most.}

The task has three purpose of downloading images of animals, which restrict participant to a specific direction,
thus, the clearness of the task is stronger than exploring task. However, the task describes a scenario of using
these images in a presentation, and hence participants must consider continuity of design style, which makes
the clearness of the task is weaker than goal-oriented task (no \emph{extracting}).

\cleardoublepage