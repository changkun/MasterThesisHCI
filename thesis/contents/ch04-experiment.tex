\section{Experiment}
\label{ch:exp}

\epigraph{We must know. We will know.}{David Hilbert}

In this chapter, the author of the thesis rationalize the process of the lab study 
based on the theory of human information behavior. 
Next, the purpose of context-given web browsing tasks is construed for the subjects.

The lab study took place during the last two weeks of November, from 14/11/2018 to 29/11/2018
in Frauenlobstrasse 7a, a faculty building of Ludwig-Maximillians-Universitaet Muenchen.
The action path data was collected by a self-developed embedded collector plugin installed in 
the mainstream browser, such as Google Chrome, on a self-provided desktop computer and a laptop.

In the lab study, the thesis selects three mainstream websites: Amazon, Medium, and Dribbble. 
These websites that cover categories for shopping, media consuming, and design brainstorming 
with design reasons (discussed later in Section \ref{sec:task-design}).
Then, the author of this thesis manually designed 35 reasonable tasks and finally selected nine 
context-given browsing tasks (three for each website, discussed in 
Section \ref{sec:task-design}) to simulate three different kinds of proposed browsing behavior,
namely \emph{goal-oriented, fuzzy and exploring behaviors}. They are defined and discussed 
later in \ref{sec:behavior}.

Each task requires participants to start from a starting page of a given website. 
The tasks do not restrict participants to using the given website only; they also allow
participants to access websites outside the domain of the landing page to help 
they complete the task 
(this information is provided to participants before participation).
Participants start browsing after they completely understand the requirements of 
each task. No interruption or question answered during the task.
If the time limit of a task exceeded, subjects can either acquire more time 
to accomplish the task or give up if they feel that the task is too difficult.

The study is designed as a \textbf{within-subject} study. Thus every participant performs all tasks. 
To eliminate the learning effect due the extensive duration of using same websites, 
a Latin square \cite{cochran1950experimental} is used in the thesis.
for the devices (desktop and laptop) and tasks participation order for the subjects.

The lab study focused on 21 participants with a mean age of 23.04 (standard deviation of 3.216, 
min=18, and max=29). Of the participants, 10 were male and 11 were female. They were recruited 
anonymously and randomly selected via a mailing list.

\subsection{Environment}

The lab study used two self-provided devices: a desktop computer and a laptop.
The reason for choosing two devices is that the study requires recording
a complete clickstream during the study.

A major issue of mobile devices is that the operating system does not
authorize the permission of allowance to collect data precisely over pages or user actions.
Although Android devices can overpass system permission to privilege, the user behavior 
between iOS and Android devices has different personalities \cite{sandoio2018android}. 
Subjects exhibit \cite{reinfelder2014androidios} abnormal awareness behavior 
regarding security and privacy issues when handling a newly provided Android device 
after they switch from an iOS device.
Therefore, to eliminate this awareness, the study focus on desktop devices,
which allows us to collects the clickstream data from browsers with plugin supports.

All modern browsers support plugin development, Google Chrome \cite{wiki2018share} has 61.7\% 
market of market share of desktop browsers, 
while Apple Safari only possesses 15.0\% of the market. 
Google Chrome is therefore dominant the desktop web browser market.

Hence, the author of the thesis decided to use Chrome to establish a plugin for collecting data.
The questionnaire after the lab study indicates the subjects' browser usage share, 
as listed in Table \ref{table:sharesubjects}. The result further support the browser selection.

\begin{table}[H]
    \small
    \centering
    \setlength{\belowcaptionskip}{10pt}
    \caption{Browser usage shares of lab study subjects}
    \begin{tabular}{ccccc}
          \toprule
        & \textbf{Google Chrome} & \textbf{Apple Safari} & \textbf{Mozilla Firefox} & \textbf{Microsoft Edge} \\
          \hline
          Number     & 11 & 5 & 3 & 2 \\
          Percentage & 52.38\% & 23.81\% & 14.29\% & 9.52\% \\
          \bottomrule
    \end{tabular}
    \label{table:sharesubjects}
\end{table}

\subsection{Browsing Behaviors}
\label{sec:behavior}

Before explaining the design justification for the context-given browsing task, 
this section presents and discusses three types of user browsing behavior: 
\textbf{goal-oriented}, \textbf{exploring} and \textbf{fuzzy}.

These three terminologies are aggregated and incorporated from behaviors that 
have been determined in former qualitative research on web browsing behavior.
These terminologies are based on the fundamental theory of 
interdisciplinary perspective information seeking behavior \cite{wilson1997information},
which was discussed in Section \ref{sec:info-seek}.
Table \ref{table:info-seek} compares the terminology differences between former research 
and this thesis.

\begin{table}[H]
    \small
    \centering
    % \setlength{\belowcaptionskip}{10pt}
    \caption{Terminology comparison of information behavior on the web}
    \begin{adjustbox}{width=\textwidth}
        \begin{tabular}{ccccc}
            \toprule
            \textbf{Author} & \textbf{Terminologies} & \textbf{Terminologies} & \textbf{Terminologies} & \textbf{Main Factors} \\
            \hline
            \cite{choo1999information} & Formal search & \makecell{Conditioned viewing; \\ Informal search} & Undirected viewing & \makecell{Psychological; demographic;\\ role-related environmental; \\source characteristics} \\
            \cite{johnson2017patterns} & \makecell{Directed browsing; \\Known-item search} & \makecell{Semi-directed browsing; \\Explorative seeking; \\``You do not know what you need''; \\Re-finding} & Undirected Browsing & Behavior \\
            \textbf{This thesis} & \textbf{Goal-oriented} & \textbf{Fuzzy} & \textbf{Exploring} & \textbf{Purpose} \\
            \bottomrule
        \end{tabular}
        \label{table:info-seek}
    \end{adjustbox}
\end{table}

To justify the terminologies, the follows combines the six qualitative activities from Ellis' Model 
\cite{ellis1989behavioural} and ``information use'' from Wilson's framework 
\cite{wilson1997information} of information behavior theory 
to represent the summarized browsing behaviors:

\paragraph{Goal-oriented behavior} \emph{occurs when a user initiates 
a visiting session on the web caused by a determined objective in a specific context}, 
such as business work, 
social communication, university study, literature research, and so on. 

Goal-oriented behavior indicates a piece of active information behavior.
Instead of \emph{formal search}, that only covers the phase of ``monitoring'' and ``extracting''
(or \emph{directed browsing} and \emph{known-item search} that covers ``browsing'' and ``differentiating'' or 
``monitoring'' and ``extracting'' respectively), 
goal-oriented browsing behavior contains the entire life cycle of human information behavior starts
from ``starting'' phase. By observing a browsing behavior, a determined ``information use'' 
can be observed and concluded.

For instance, a college student intentionally need a latest lecture slide (\emph{information use} observed), 
the student then opens web browser, access college website (\emph{starting}) and navigates to the lecture homepage 
(\emph{chaining}, \emph{browsing}, and \emph{differentiating}).
Finally, the student exit browsing after download the slides (\emph{monitoring} and \emph{extracting}).

\paragraph{Exploring behavior} \emph{occurs when a user initiates browsing session 
aimlessly with no clear observed extracting or information use during the session, 
the person greedily or breadth-first consumes and the content on the Web without 
any information extracting and information use}, such as media consuming, learning 
before using and so on.

Exploring browsing behavior indicates opposite behavior from 
goal-oriented browsing behavior. A more formal description of exploring behavior using Ellis' model, 
would note that the behavior represents ``chaining'' and ``browsing''
without ``differentiating'' and ``extracting'' from ``starting'' while information seeking.

For instance, a person who accesses an unknown utility web application (\emph{starting}), 
may explore the functions one by one as well as 
what they can do while using the application (\emph{chaining} and \emph{browsing}).

\paragraph{Fuzzy behavior} \emph{occurs when a user initiates a visiting session for information use
with non-systematic and incomplete prior knowledge that may involve ongoing browsing ongoing to update 
the framework of knowledge until final acquisition or abandon.}

Fuzzy behavior in browsing behavior is in between goal-oriented and exploring behaviors.
Instead of only ``chaining'' and ``browsing'' from ``starting'', fuzzy behavior also engages
``differentiating'' or ``monitoring'' while information seeking.

For instance, a researcher may have heard a new technique proposed in another scientific field 
that may influence their research. That person may then 
opens a search engine (\emph{starting} and \emph{chaining}) to seek (\emph{browsing}) 
existing (\emph{differentiating}) follow-up research (\emph{monitoring}). 
The browsing may end without information use because the technique is irrelevant 
to their research.

\paragraph{Remark} Table \ref{table:ellis} illustrates the existence of activities of three
forms of browsing behavior. 
Note that ``information needs'' is not suggested in Wilson's theory \cite{wilson1981user} 
because they can not be clearly observed before information seeking but sometimes
may be observed after information use. Therefore information need is not considered in these terminologies.

\begin{table}[H]
    \small
    \centering
    % \setlength{\belowcaptionskip}{10pt}
    \caption{Existence of activities from Ellis' Model and information use in 
    goal-oriented, exploring and fuzzy browsing behavior.
    The ``exist'', in the table, represents the existence of which activities contributes in which pattern.
    Information need is ``N/A'' because Wilson's theory does not suggest using information need to
    define terminology because it cannot be observed before information use.
    }
    \begin{adjustbox}{width=\textwidth}
        \begin{tabular}{ccccccccc}
            \toprule
            \multicolumn{1}{c}{\multirow{2}{*}{\textbf{Behaviors}}}  & \multicolumn{1}{c}{\multirow{2}{*}{\textbf{Information Need}}} & \multicolumn{6}{c}{\textbf{Information Seeking}}                                    & \multicolumn{1}{c}{\multirow{2}{*}{\textbf{Information Use}}} \\ \cline{3-8}
            \multicolumn{1}{c}{}                                     & \multicolumn{1}{c}{}                                  & \textbf{Starting} & \textbf{Chaining} & \textbf{Browsing} & \textbf{Differentiating} & \textbf{Monitoring} & \textbf{Extracting} & \multicolumn{1}{c}{}  \\
            \hline
            Goal-oriented                                            &   N/A                                                 &     Exist         &      Exist        &   Exist           &        Exist               &       Exist           &        Exist          &      Exist              \\
            Fuzzy                                                &   N/A                                                 &     Exist         &      Exist        &   Exist           &        Exist               &       Exist           &                     &                       \\
            Exploring                                                    &   N/A                                                 &     Exist         &      Exist        &   Exist           &                          &                     &                     &                       \\
            \bottomrule
        \end{tabular}
        \label{table:ellis}
    \end{adjustbox}
\end{table}

\subsection{Tasks Design}
\label{sec:task-design}

The author of the thesis designed 35 browsing tasks after conducting a pilot study. 
Nine tasks were selected for three websites: Amazon.com, Medium.com, and Dribbble.com 
because of the following reasons:

\begin{enumerate}
    \item These three websites all have tasks that correspond to the three types of browsing behavior;
    \item Each of the tasks can be finished in around 5 to 10 minutes according to the measurement of pilot study;
    \item All these websites are mainstream websites that do not require 
        significant professional domain knowledge to use.
\end{enumerate}

In addition, the unselected tasks are listed in Appendix \ref{appendix:unselected}.

\subsubsection{Tasks of Goal-oriented Behavior}

The author designed and selected an appropriate goal-oriented task for selected websites.
Each task is designed with three designed information needs as
the justification for information use.

\paragraph{Amazon.com} \emph{Assume your smartphone was broken and you have 1,200 euros 
    as your budget. You want to buy an iPhone, a protection case, and a wireless 
    charging dock. Look for these items and add them to your cart.}

This task initiates from the homepage of Amazon (\emph{starting} and \emph{chaining}), 
and contains three determined objective since 
a subject is required to add three specific items to the cart (\emph{information use}). 
There are a few hidden considerations behind the task (\emph{browsing} and \emph{differentiating}),
which makes the task more realistic (\emph{monitoring} and \emph{extracting}): 
a) There is a budget for this task, which requires subjects to consider the
price of items instead of simply adding the first recommended item to cart. 
b) the starting page is amazon.com instead of amazon.de. This decision requires
subjects to consider the exchange rate between U.S. dollars and euros for budgeting.
c) There are some items cannot be shipped to Germany (the study took place in Germany).
Subjects cannot add these items to the cart and should find other alternatives.

\paragraph{Medium.com} \emph{Assume you are making plans for your summer vacation. 
        You want to visit Tokyo, Kyoto, and Osaka. You want to find out what kind of experience other people have had 
        when traveling to these three places in Japan. Your task is to find three posts on 
        traveling tips regarding these cities. Elevate a post if it is one of your choices.}

This task contains three determined purposes because there are three fixed traveling 
destination (\emph{extracting} and \emph{information use}).
The task also involves a few considerations that increase the required interaction between 
the task to subjects:
a) The website only offers an English version. Some Japanese characters may appear in an article.
Thus, a translation website may be used during the study (\emph{starting} and \emph{chaining}).
b) An article may contain numerous nouns, such as toponyms. Search engines may used during the study (\emph{browsing}).
c) Articles, that require a membership to access, cannot be elevated (\emph{differentiating}). 

\paragraph{Dribbble.com} \emph{You are hired at a cloud computing startup company. You receive assignment to 
        design the logo of the company. Search for existing logos for inspiration and 
        download three candidate logos that you like the most.}

The task also has three determined purpose because subjects are required to download 
three candidate trademarks (\emph{extracting} and \emph{information use}).
During the participation, subjects must take a few implicit facts in to account:
a) Subjects who unfamiliar with the term ``Cloud Computing'' must visit other explainations 
to determine the vision and mission of this type of company (\emph{starting}). 
Subjects who are already familiar with the term still need to compare the designs 
made by other competitors (\emph{chaining}, \emph{browsing} and \emph{differentiating}).
b) Subjects should aware that some of the designs shared on the website are not suitable 
for trademark or icon design (\emph{monitoring}).

\subsubsection{Tasks of Exploring Behavior}

Exploring tasks simply do not provide any deterministic objective,
and all websites have a exploring task that is designed for subjects.

\paragraph{Amazon.com} \emph{Look for a product category that you are interested in and start browsing. 
        Add three items that you would like to buy to your cart.}

Although the task do not require any specific items from the subjects, 
the task remains to have three different purposes because participants must 
add three items to the cart. This task is aimless because: 
all the tasks are not specifically informed to participants. 
They either do not have the needs to buy items or 
had needs of buy a specific category but do not have a product candicate yet.
In any case, the description of the task request participants to start from a product category (\emph{starting} and \emph{chaining}), 
which avoids goal-oriented buying of a specific product.

\paragraph{Medium.com} \emph{Visit a category you are interested in and elevate 
three posts that you like.}

This task has a similar reason to the one as discussed in Amazon.com's exploring task (\emph{starting} and \emph{chaining}). 
Medium is a media website. Hence, visiting a specific article that read before participation is relatively difficult 
because all the content that is showed to users is updated daily. 
Thus, this task can be considered to be an exploring task.

\paragraph{Dribbble.com} \emph{Explore Dribbble and download the three images 
you like the most while you browse.}

Dribbble illustrates designs by using the image gallery (\emph{starting} and \emph{chaining}). 
The major difference between Dribbble and Google Image Search
is that Dribbble is a user-centered content aggregation website, while Google Image Search 
is a simple content aggregation engine.
Hence, there will be two different interactions in Dribbble: exploring designs based on 
keywords and categories or exploring designs based on users. 
The latter helps its user to finding similar design style.
The task is aimless because the task simply describes nothing and lets participants explore 
their preferences for design styles.

\subsubsection{Tasks of Fuzzy Behavior}

Each of the selected websites also has a fuzzy task, and there are three major goals for each task that act
as control conditions to subjects' action paths in the experiment.

\paragraph{Amazon.com} \emph{You want to buy a gift for your best friend as a birthday present.
        Add three items to your cart as candidate.}

The clarity of the task is stronger than the exploring task but is weaker than 
the goal-oriented task, because
the task restricts participants from adding items for a specific purpose (birthday present) 
but does not point to any specific product (no \emph{extracting}).

\paragraph{Medium.com} \emph{Assume you have an occasion to visit China for business. 
        You are free to travel to China for a week and want to make a travel plan for that time frame.. 
        Your task is to determine what kind 
        of experiences other people have had when visiting to secondary cities or towns in China, then decide 
        on three cities you want to visit (excluding Beijing, Shanghai, Guangzhou, and Shenzhen). 
        Elevate a post if it helped you to decide.}

The clearness of the task is stronger than exploring the task because it asks a participant 
to explore a non-deterministic direction of looking for secondary cities (no \emph{extracting}).
However, the clearness of the task is weaker than the goal-oriented task because 
the secondary cities described in Medium's user posts are unclear, and participants are 
supposed to make decisions themselves.
Furthermore, this task pertains to traveling around  China for a week. 
Cities cannot be randomly
selected because making travel plans requires consideration of a city's geographic 
location.

\paragraph{Dribbble.com} \emph{You are preparing a presentation and need one picture for each of these animals: 
    cat, dog, and ant. Download the three pictures you like the most.}

The task has three purposes of downloading images of animals, which restricts participant to a specific direction.
Thus, the clearness of the task is stronger than the exploring task. 
However, the task describes a scenario of using
these images in a presentation. Hence participants must consider the continuity of 
the design style, which makes the clearness of the task weaker than 
the goal-oriented task (no \emph{extracting}).

\cleardoublepage