\section{Lab Study}
\label{ch:exp}

The lab study took place during the last two weeks of November, from 14/11/2018 to 29/11/2018
in Frauenlobstrasse 7a, a faculty building of Ludwig-Maximillians-Universitaet Muenchen.
Data was collected from the mainstream browser Google Chrome on a self provided 
desktop computer and a laptop. All 21 subjects are recruit anonymously and randomly.

The following of this chapter, we present the process of our lab study then construe the purpose 
of context given web browsing tasks for our subjects.

% - 介绍实验的具体细节,包括所有的实验任务
% - 为搜集任务的设计进行辩护,为什么任务是合理的,clickstream 和 searchstream 之间有什么区别
% - 在这里陈述搜集到的实验结果,包括 subjects 的总体数据

\subsection{Pilot Study}

Initially, we developed a web crawler that downloaded the entire medien computer science website 
\footnote{\url{https://medien.ifi.lmu.de}}
.... TODO: discuss how we go here.

\subsection{Study}

In our lab study, we selected three mainstream websites, Amazon/Medium/Dribbble 
that covers categories for shopping, media consuming and design brainstorming. 
We manually designed nine reasonable (discussed in section \ref{sec:select-tasks})
context-given browsing tasks (three for each website) in total to our subjects.
Each task requires participant start from the homdpage of a given website, and
all tasks do not restrict participants use the given website, but also allow they 
access websites outside the landing page to help they complete the task.
Participants start browsing after they completely understand the requirements of each task.
To simulate a real case, there is no interruption and question answering during the task.

\subsubsection{Environment}

The lab study uses two self provided device: a desktop computer and a mobile laptop.
The reason of choose two morphology of computing device is our study requires recording
a complete clickstream of during the study.

A major issue of mobile devices is the operating system installed in mobile phone does not
open the permission to allow us to collect data precisely over pages. Though Android device
can overpass system permission to privilge, the user behavior between iOS and Android device
is still completely different with different models. Subjects shows abnormal behavior when they
use a newly provided device. Therefore we stick our study environment to desktop devices, 
which empower us easily collects the clickstream data from plugin-supported browsers.
To eliminate the learning effect of computing devices, we use Latin square \cite{cochran1950experimental} 
for the participation order of desktop and laptop to our subjects.

Considering the usage share of all browsers on the market, till the month we runs our study (November),
Google Chrome wins the usage share of desktop browsers with 61.75\% shares, and Apple Safari only shares
15.12\% of the market. Clearly, Google Chrome dominant the desktop web browser. 

Therefore, we decide to use Chrome to establish our plugin of data collection.
A quesionaire after our lab study indicates the browsers usage share of all subjects, 
as shown in Table \ref{table:sharesubjects}, which further supports our decision of 
browser selection.

\begin{table}[H]
    \small
    \centering
    \setlength{\belowcaptionskip}{10pt}
    \caption{Browser Usage Shares of Lab Study Subjects}
    \begin{tabular}{ccccc}
          \toprule
        & \textbf{Google Chrome} & \textbf{Apple Safari} & \textbf{Mozilla Firefox} & \textbf{Microsoft Edge} \\
          \hline
          Number     & 11 & 5 & 3 & 2 \\
          Percentage & 52.38\% & 23.81\% & 14.29\% & 9.52\% \\
          \bottomrule
    \end{tabular}
    \label{table:sharesubjects}
\end{table}

\subsubsection{Context-given Websites}

In our study, we gathered various websites covering most of the categories that people do
on the Internet: Social Networks, Shopping, Email, Media Consuming, Search, Production 
and etc \cite{lori2018internet, lori2017internet}. All selected website are listed 
in Table \ref{table:websites}.

\begin{table}[H]
    \small
    \centering
    \setlength{\belowcaptionskip}{10pt}
    \caption{Browser Usage Shares of Lab Study Subjects}
    \begin{tabular}{ccccc}
          \toprule
        & \textbf{Google Chrome} & \textbf{Apple Safari} & \textbf{Mozilla Firefox} & \textbf{Microsoft Edge} \\
          \hline
          Number     & 11 & 5 & 3 & 2 \\
          Percentage & 52.38\% & 23.81\% & 14.29\% & 9.52\% \\
          \bottomrule
    \end{tabular}
    \label{table:sharesubjects}
\end{table}

\subsubsection{Designed Tasks}
\label{sec:select-tasks}

Before we explain the design reason of our context-given browsing task, we first present
and discusses the common three types of daily browsing.

\textbf{Goal-oriented Task}: Typically, an Internet user opens the browser for information retrieve
for bussiness work, school study, literature research and reasons that made the person has
clear purpose.

\textbf{Exploring Task}: A user visits the browser aimlessly, explores and consumes the content
lies on the Web.

\textbf{Fuzzy Task}: 

\cleardoublepage