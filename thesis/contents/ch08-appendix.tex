\part*{Appendix}
\appendix
\addcontentsline{toc}{part}{Appendix}
\fancyhead[LE,RO,LO,RE]{} % No headline on top of each page

All resources relates to the thesis are open source, 
they can be found publicly in:

\begin{itemize}
    \item Thesis homepage: \url{https://changkun.us/thesis/};
    \item GitHub repostory: \url{https://github.com/changkun/MasterThesisHCI/}.
\end{itemize}

All related text, picture and video content are licensed under a 
Creative Commons Attribution-NonCommercial-ShareAlike 4.0 International 
License\footnote{\url{http://creativecommons.org/licenses/by-nc-sa/4.0/}}.
The other parts of the thesis (such as program source code) are licensed 
under a MIT Public License
\footnote{\url{https://github.com/changkun/MasterThesisHCI/blob/master/LICENSE}}.

\section{Content of enclosed USB}
\label{appendix:a}

\begin{enumerate}
    \item $/documents/$ - TODO
\end{enumerate}


\section{Tasks and Questionnaire in Lab Study}
\label{appendix:b}

\subsection{Phase 1: Browsing Task}

This section approximately takes 80 minutes.

In this study, you are asked to accomplish a series of tasks provided in the table below.
Please read the following tips carefully before you do the task \footnote{The order of the tasks
are rearranged through Latin square, this section only illustrate one possible order of tasks}.

\begin{enumerate}
    \item \textbf{Please start from the given starting page.} You can then visit any other page. 
          For instance, if you find a task too difficult, you can visit any other websites 
          that help you accomplish the task (e.g. Google as a search engine), but you 
          should only use the browser.
    \item The tasks are designed to take \textbf{5~10 minutes}. Do not feel stressed if you spend 
          more time because you have 80 minutes in total to \textbf{do the 9 tasks}. You will 
          be notified if you spend more than 10 minutes on a task. You can decide to go to 
          the next task or spend some to accomplish the unfinished task. 
    \item \textbf{Close the browser before you start working on the next task.}
    \item \textbf{Unfortunately, questions cannot be answered while doing the tasks.
          Please ask them before starting a task if something is not clear. }
\end{enumerate}

\subsubsection{Task Group 1: Amazon.com}

\textbf{Task Category: Shopping}

\begin{enumerate}
    \item Assume your smartphone was broken and you have 1200 euros 
          as your budget. You want to buy an iPhone, a protection case, and a wireless 
          charging dock. Look for these items and add them to your cart.

          \textbf{Requirement to Finish}: Click ``Proceed to checkout'' when you finished, exit the browser when you see the ``sign in'' page.
    \item You want to buy a gift for your best friend as a birthday present.
          Add three items to your cart as candidate.
    
          \textbf{Requirement to Finish}: Click ``Proceed to checkout'' when you finished, exit the browser when you see the ``sign in'' page.
    \item Look for a product category that you are interested in and start browsing. Add three items to your cart that you would like to buy. 
           
          \textbf{Requirement to Finish}: Clicked ``Proceed to checkout'' when time is up, exit the browser when you see the ``sign in'' page.
\end{enumerate}

\textbf{How difficult was the task? (1~5, 1 means very easy, 5 means very difficult)}

$\rule{1cm}{0.15mm}$, $\rule{1cm}{0.15mm}$, $\rule{1cm}{0.15mm}$

\subsubsection{Task Group 2: Medium.com}

\textbf{Task Category: Media}

\begin{enumerate}
    \item Assume you were making plans for your summer vacation. You want to visit Tokyo, Kyoto, and Osaka. 
          You want to find out what kind of experience other people made 
          when traveling to these three places in Japan. Your task is to find three posts 
          for traveling tips regarding these cities. Elevate a post if it is one of your choices.

          \textbf{Requirement to Finish}: Write down three tips.
          Close the browser when you are finished.

          \item Assume you got an occasion to visit China for business. You are free to travel to China for a week. 
          You want to make a travel plan for touring China within a week. Your task is to find out what kind 
          of experience other how people made when going to secondary cities or towns in China, then decide 
          on three cities you want to visit (excluding  Beijing, Shanghai, Guangzhou, and Shenzhen). 
          Elevate if a post helped you make a decision. 

          \textbf{Requirement to Finish}: Write down the names of the cities you decided. 
          Close the browser when you are finished.

    \item Visit a category you are interested in and elevate the post you like. 
    
          \textbf{Requirement to Finish}: Close the browser when time is up.
\end{enumerate}

\textbf{How difficult was the task? (1~5, 1 means very easy, 5 means very difficult)}

$\rule{1cm}{0.15mm}$, $\rule{1cm}{0.15mm}$, $\rule{1cm}{0.15mm}$

\subsubsection{Task Group 3: Dribbble.com}

\textbf{Task Category: Design}

\begin{enumerate}
    \item You are hired to a Cloud Computing startup company. You get an assignment to 
    designing the logo of the company. Search for existing logos for inspiration and 
    download three candidate logos you like the most.

    \textbf{Requirement to Finish}: Close the browser when you finished the download.

    \item You are preparing a presentation and need one picture for each of these animals: 
    cat, dog, and ant. Download the three pictures you like the most.

    \textbf{Requirement to Finish}: Close the browser when you finished the download.

    \item Explore dribbble and download images you like the most while you browse.
    
    \textbf{Requirement to Finish}: Close the browser when you finished the download.

\end{enumerate}

\textbf{How difficult was the task? (1~5, 1 means very easy, 5 means very difficult)}

$\rule{1cm}{0.15mm}$, $\rule{1cm}{0.15mm}$, $\rule{1cm}{0.15mm}$

\subsection{Phase 2: Questionnaire}

This section approximately takes 10 minutes.

\begin{enumerate}
    \item Age: $\rule{1cm}{0.15mm}$
    \item Gender: Female / Male
    \item What is your study program or occupation? 
    \item What are the websites that you access mostly? List your top-5 (max 10, including private use).
    \item What do you usually do  when you access these websites? Shortly answer your case for all the websites you listed in above and name  two common reasons, ordered by frequency.
    (For example, for YouTube, the most common reason could be ``Just for fun'', the second most common reason ``Looking for tutorial''. Then write as ``Mostly for fun, sometimes for learning'' below. )
    \item Do you use bookmarks to save webpages that you have found through a search engine? If so, why? 
    \item Which browser do you use mainly on your PC or Mac? 
          Chrome / Safari / IE / Microsoft Edge / Firefox / Others, the name is: $\rule{1cm}{0.15mm}$
    \item Would you like to participate in a follow-up study? The study will ask you to install a browser plugin for a week which anonymously records your browsing history. 
          Yes / No
    \item Do you have any feedback on this questionnaire?
\end{enumerate}

\subsection{Unselected Tasks}
\label{appendix:unselected}

This section lists all designed tasks but unselected to lab study.

\subsubsection{Goal-oriented Task}

\begin{enumerate}
\item \textbf{www.github.com}: You are comparing three most popular frontend desktop frameworks: Electron / NW.js / ReactNative Desktop. Your goal is to find out the latest release download link.
\item \textbf{www.medien.ifi.lmu.de}: You are a fresh medieninformatik student major in HCI program. You wants to find out recommended first semester study plan provided by the program, then select "Human-Computer Interaction II" opened in WS18/19 and check previous "Human-Computer Interaction I" opened in SS18 and SS17.
\item \textbf{www.en.uni-muenchen.de}: You are a international student who want to apply economics program  for your master study at LMU. Find the page for application requirement.
\item \textbf{www.ielts.org}: You live in Munich, you want to participate to IELTS test next year on Feburary. Looking for the entrace to register the examination. You must keep seeking and stop when you selected the first track of Feburary test.
\item \textbf{www.bloomberg.com}: You somehow heared about Bloomberg reported a news about China use tiny chips infiltrate U.S companies. You wants to find the article.
\item \textbf{www.reddit.com}: You are a fan of Marvel comics, you want to view some spoilers regarding a comming moive "The Avengers 4". Find latest three post that spoilers The Avengers 4.
\item \textbf{www.facebook.com}: You are a facebook user, and you have a wide social. However you don't wants to see parenting information in your timeline, you wish to turn them off for a year from your timeline; then recently you start interested in ping pong, you want to join a related local group.
\item \textbf{www.twitter.com}: You lost your phone and phone number, and you bought a new one. However the old phone number was registered in your twitter account, you want to change it for your account safety. Please find the entrace to change your phone number and password. Then you becomes curious on twitter's settings. You want to know how twitter use your data and prevent twitter collect your data.
\item \textbf{www.youtube.com}: You want to be a Youtuber. You wants to know how to earn money from making videos, and what should you concern when you publishing a video.
\item \textbf{www.google.com}: You can't access your gmail. You want to findout whether gmail are current malfuntioning or not. Contact instance messaging support.
\end{enumerate}

\subsubsection{Fuzzy Task}

\begin{enumerate}
\item \textbf{www.github.com}: You were a senior developer. Your boss wants you write a report regarding the tends of current development techniques. You want to find the most three popular (top-3 stars) web backend Go frameworks and access their repository,  write their name down on a paper when you decided.
\item \textbf{www.medien.ifi.lmu.de}: You are a fresh medieninformatik student. You wants to select three lectures, one seminar and one practicum for your study in WS18/19.
\item \textbf{www.arxiv.org}: Find the most recent published a overview paper for these three topics respectively: affective computing, convolutional neural networks, distributed consistency algorithm.
\item \textbf{www.google.com}: You want to know how google profiling you based on your history. Find your personality profile that created by Google.
\item \textbf{www.bloomberg.com}: You want to find the relevant news regarding the progress of China use tiny chips infiltrate U.S companies.
\end{enumerate}

\subsubsection{Exploring Task}

\begin{enumerate}
\item \textbf{www.github.com}: Browsing github and select three github repository your most interested in.
\item \textbf{www.medien.ifi.lmu.de}: Browsing the website until time is up.
\item \textbf{www.en.uni-muenchen.de}: Browsing the website until time is up.
\item \textbf{www.ielts.org}: Browsing the website to see what you can do except register to examination.
\item \textbf{www.bloomberg.com}:  Browsing the website until time is up.
\item \textbf{www.reddit.com}:  Browsing the website until time is up.
\item \textbf{www.facebook.com}:  Browsing the website until time is up.
\item \textbf{www.twitter.com}:  Browsing the website until time is up.
\item \textbf{www.youtube.com}:  Browsing the website until time is up.
\item \textbf{www.arxiv.org}: Browsing the website for categories you interested in until time is up.
\item \textbf{www.google.com}: Browsing google until time is up.
\end{enumerate}

\section{Raw Data Illustration}
\label{appendix:c}

\subsection{Subjective Difficulty Score from Lab Study}

\begin{table}[H]
      \small
      \centering
      \setlength{\belowcaptionskip}{10pt}
      \caption{Subjective task difficulty from lab study}

      \begin{tabular}{cccc}
            \toprule
            \textbf{Subject ID} & \textbf{Amazon.com} & \textbf{Medium.com} & \textbf{Dribbble.com} \\
            \hline
            1 & 2, 1, 2 & 2, 4, 1 & 2, 3, 2 \\
            2 & 2, 2, 1 & 2, 3, 1 & 1, 5, 1 \\
            3 & 3, 2, 2 & 2, 5, 3 & 3, 1, 3 \\
            4 & 3, 4, 2 & 2, 5, 2 & 3, 3, 2 \\
            5 & 2, 1, 3 & 3, 5, 3 & 2, 1, 3 \\
            6 & 2, 2, 1 & 3, 4, 1 & 1, 3, 2 \\
            7 & 3, 4, 2 & 3, 5, 3 & 4, 3, 2 \\
            8 & 1, 1, 1 & 3, 5, 2 & 2, 1, 1 \\
            9 & 2, 3, 2 & 2, 5, 2 & 3, 1, 1 \\
            10 & 1, 3, 2 & 2, 3, 2 & 2, 3, 3 \\
            11 & 2, 2, 3 & 1, 4, 5 & 1, 2, 3 \\
            12 & 3, 2, 1 & 3, 4, 1 & 3, 2, 2 \\
            13 & 4, 1, 3 & 5, 4, 2 & 2, 2, 1 \\
            14 & 2, 2, 2 & 2, 3, 1 & 2, 2, 1 \\
            15 & 5, 1, 3 & 2, 4, 1 & 4, 2, 3 \\
            16 & 1, 2, 1 & 1, 3, 1 & 1, 1, 1 \\
            17 & 3, 1, 1 & 3, 4, 3 & 2, 2, 3 \\
            18 & 2, 2, 1 & 2, 3, 1 & 3, 2, 2 \\
            19 & 3, 2, 2 & 2, 2, 1 & 1, 1, 2 \\
            20 & 1, 3, 2 & 3, 5, 1 & 2, 3, 2 \\
            21 & 3, 3, 2 & 3, 5, 4 & 2, 3, 5 \\
            \bottomrule
      \end{tabular}
      \label{table:diff}
\end{table}
